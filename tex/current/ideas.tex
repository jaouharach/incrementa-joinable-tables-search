
\section{Ideas and Questions}
\label{sec:ideas}

\subsection{Ideas}
\begin{enumerate}
	\item The dataset discovery process is more likely to be iterative. At the beginning the user forms a general query for his information need. Retrieved datasets help the user better understand his need and hence better reformulate his query in the next iteration.
	
	\item Guarantee interactive-level response time.
	\item The relationship that we aim to capture between datasets oftentimes defines the structure of the query that search engine will accept.
	\item speed up the data science workflow.
	
	
	\item Similarity measures:
	\begin{enumerate}
		\item Text data: Hammin distance, Jaccard similarity, Jaccard containment ...
		\item Numerical data: Correlation, cosine distance, euclidean distance ...
		\item Binary data: ...
		\item TF-IDF, Okapi BM25, LDA topic modeling
	\end{enumerate}
	\item Order based similarity measures (image retrieval)?
\end{enumerate}

\subsection{Questions}
\begin{enumerate}
	\item If the query time is too small how can the user observe changes in query results as they arrive incrementally? 
	
	\item [+] The user can observe changes in query results because when querying over Terabytes (or even Petabytes of data) the query time will increase significantly.
	
	
\end{enumerate}