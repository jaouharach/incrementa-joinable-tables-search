\section{Introduction}
\label{sec:introduction}

\begin{itemize}
	\item \textbf{Use case 1: Keyword Query}\\
	 A data scientist wants to analyze the impact of food cost inflation on food consumption. Initially  The user decides to start the search with the keywords \{"food", "consumption"\} and $k = 10$ . The search engine returns Table \ref{tab:q1-result1} which contains data from year 1990 to 2009 about {\it "Per Capita Consumption of Principal Foods (in pounds)"}. The user decides to keep Table \ref{tab:q1-result1} for the study and continue to search for other relevant tables. %k = 10
	
	\item \textbf{Use case 2: Join Query}
	Table \ref{tab:q1-result1} is a good first result as it contains a complete list of the main food types, however the result lacks information on food prices. For that the use perform a join query on the food column to explore other tables that may have information about food prices for the years 1990 to 2009. \\*
	\textbf{Attempt 1:} To speed up search the user submits the first query $Q_{1} =$ (Table \ref{tab:q1-result1}, Join column : "Food", $k = 10$). The search engine returned 775 tables. However, after skimming through the list of returned tables nothing seemed relevant to the user. 
	
	\textbf{Attempt 2:} The user decides to increase $k$ to get more results from the search engine. He/she submits a second query $Q_{2} =$ (Table \ref{tab:q1-result1}, Join column : "Food", $k = 20$). This time the search engine returned 161 tables, because the number of results is big the user could notice Table \ref{tab:q1-result2} ranked at position 55.
	
	\textbf{Attempt 3:} For the last attempt the user gave up on getting any fast meaningful result so he/she decide to increase $k$ significantly in hope that a relevant table will appear in the list of results. He/she submits $Q_{3}$ to the search engine.$Q_{3} =$ (Table \ref{tab:q1-result1}, Join column : "Food", $k = 200$). Finally and after several attempts, the search engine returns Table \ref{tab:q1-result2} at position $11$ which contains information on food prices from the 2007 WIC program. % k = 100 \\

	\begin{table*}[t]
		\setlength{\tabcolsep}{0.9\tabcolsep}
		\begin{tabular}{ |l|l|l|l|l|l|l|l|  }
			\hline
			Food&	1990&	1995&	2000&	2002&	2004&	2005&	2009\\
			\hline
			Wheat flour& 	135.9& 	140.0& 	146.3& 	136.8& 	134.3& 	134.1& 	134.6\\
			Vegetables& 	386.0& 	407.6& 	423.0& 	411.8& 	422.8& 	415.4& 	390.9\\
			Veal& 	0.9& 	0.8& 	0.5& 	0.5& 	0.4& 	0.4& 	0.3\\
			Turkey& 	13.8& 	13.9& 	13.7& 	14.0& 	13.4& 	13.1& 	13.3\\
			Tree nuts& 	2.45& 	1.94& 	2.57& 	3.24& 	3.62& 	2.7& 	3.7\\
			Rice (milled basis)& 	15.8& 	17.1& 	18.9& 	19.5& 	20.4& 	21.0& 	21.2\\
			Red meats2,3,4& 	112.2& 	113.6& 	113.7& 	114.0& 	112.0& 	110.0& 	105.7\\
			Poultry2,3,4& 	56.2& 	62.1& 	67.9& 	70.7& 	72.7& 	73.6& 	69.4\\
			... & ...& ...& ...& ...& ...& ...& ...\\
			\hline
			
		\end{tabular}
		\vspace{.2cm}
		\caption{ U.S. Economy and the Federal Budget Economy: Per Capita Consumption of Principal Foods \label{tab:q1-result1}}
	\end{table*}
	
	\begin{table*}[!ht]
	\setlength\extrarowheight{2pt} % for a bit of visual "breathing space"
		\begin{tabular}{ |l|p{8cm}|l|p{4cm}|  }
			\hline
			\textbf{Food item}&	\textbf{Retail sales database selection criteria}&	\textbf{Units}&	\textbf{Price per unit (inflated to FY06)}\\
			\hline
			
			Yogurt& 	Quart sized containers and larger. Plain, vanilla, and fruit flavors& 	qt& 	2.068\\\hline
			Whole-grain& bread 	Wheat or grain bread& 	lb& 	1.422\\\hline
			Whole& 	Fresh dairy milk only,1/2gallon or gallon containers. Reduced fat includes skim milk and milk identified as 2\% or lower milk fat& 	qt& 	0.767\\\hline

			Tuna& 	Chunk light, canned& 	oz& 	0.101\\\hline
		
			Peanut butter& 	All forms and varieties. Not mixed with jelly& 	oz& 	0.094\\\hline
			Brown rice& 	Instant or regular& 	lb& 	1.178\\\hline
			...& 	...& 	...& 	...\\
			\hline
			
		\end{tabular}
		\vspace{.2cm}
		\caption{Federal Register | Special Supplemental Nutrition Program for Women, Infants and Children (WIC), 2007\label{tab:q1-result2}}
	\end{table*}
	
	
\end{itemize}
